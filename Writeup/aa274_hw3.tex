\documentclass[12pt]{article}
\usepackage{fullpage,graphicx,psfrag,url,ar,rotating,float}
\usepackage[small,bf]{caption}
\usepackage{amsmath,amssymb,enumitem,bbm,subcaption,multirow}
\setlength{\captionmargin}{20pt}
\setlist[enumerate]{label=(\roman*)}

\title{\Large{AA274 (Winter 2017-18): Problem Set 3}}
\author{Anqi Fu}

\begin{document}
\maketitle

\section{Extended Kalman Filter (EKF)}
\begin{enumerate}
	\item See submitted code. Let the state variables be $\mathbf{x}(t) = (x(t), y(t), \theta(t))$ and the control variables be $\mathbf{u}(t) = (V(t), \omega(t))$. The differential drive model is
	\begin{align*}
		\dot x(t) &= V(t)\cos(\theta(t)) \\
		\dot y(t) &= V(t)\sin(\theta(t)) \\
		\dot \theta(t) &= \omega(t),
	\end{align*}
	which we can discretize by assuming a zero-order hold on the control, i.e. $\mathbf{u}(t)$ is constant over a time interval of length $dt$. Then we can integrate to get
	\[
		\theta_t - \theta_{t-dt} = \int_{t-dt}^t \dot \theta(s)ds = \int_{t-dt}^t \omega(s)ds = \omega_t\int_{t-dt}^tds = \omega_tdt.
	\]
	If $\omega(t) = 0$ over the interval, $\theta(t)$ remains constant, so our coordinates can be found easily by integrating
	\begin{align*}
		x_t - x_{t-dt} &= \int_{t-dt}^t \dot x(s)ds = \int_{t-dt}^t V(s)\cos(\theta(s))ds = V_t\cos(\theta_{t-dt})dt \\
		y_t - y_{t-dt} &= \int_{t-dt}^t \dot y(s)ds = \int_{t-dt}^t V(s)\sin(\theta(s))ds = V_t\sin(\theta_{t-dt})dt.
	\end{align*}
	With a slight abuse of notation, We may thus write
	\[
		\mathbf{x}_t = \left(\begin{array}{c}
		x_t \\
		y_t \\
		\theta_t\end{array}\right) = \left(\begin{array}{c}
		x_{t-1} + V_t\cos(\theta_{t-1})dt \\
		y_{t-1} + V_t\sin(\theta_{t-1})dt \\
		\theta_{t-1} + \omega_tdt
		\end{array}\right) = g(\mathbf{x}_{t-1}, \mathbf{u}_t)
	\]
	and the Jacobians of $g$ with respect to $\mathbf{x}$ and $\mathbf{u}$ are, respectively,
	\[
		G_x = \left(\begin{array}{ccc}
		1 & 0 & -V_t\sin(\theta_{t-1})dt \\
		0 & 1 & V_t\cos(\theta_{t-1})dt \\
		0 & 0 & 1
		\end{array}\right) \quad \mbox{and} \quad
		G_u = \left(\begin{array}{cc}
		\cos(\theta_{t-1})dt & 0 \\
		\sin(\theta_{t-1})dt & 0 \\
		0 & dt
		\end{array}\right).
	\]
	If $\omega(t) \neq 0$, we exploit the fact that $\dot \theta(t) = \omega(t)$ to get
	\begin{align*}
		x_t - x_{t-dt} &= \int_{t-dt}^t V(s)\cos(\theta(s)) \frac{\dot \theta(s)}{\omega(s)}ds = \frac{V_t}{\omega_t}\int_{t-dt}^t \cos(\theta(s)) \dot \theta(s)ds \\
		&= \frac{V_t}{\omega_t} (\sin(\theta_t) - \sin(\theta_{t-dt})) = \frac{V_t}{\omega_t}(\sin(\theta_{t-dt}+ \omega_tdt) - \sin(\theta_{t-dt})) \\
		y_t - y_{t-dt} &= \frac{V_t}{\omega_t}\int_{t-dt}^t \sin(\theta(s))\dot \theta(s)ds = \frac{V_t}{\omega_t}(-\cos(\theta_{t-dt} + \omega_tdt) + \cos(\theta_{t-dt}))
	\end{align*}
	Our discrete model becomes
	\[
		\mathbf{x}_t = \left(\begin{array}{c}
		x_t \\
		y_t \\
		\theta_t\end{array}\right) = \left(\begin{array}{c}
		x_{t-1} + \frac{V_t}{\omega_t}(\sin(\theta_{t-1}+ \omega_tdt) - \sin(\theta_{t-1})) \\
		y_{t-1} - \frac{V_t}{\omega_t}(\cos(\theta_{t-1} + \omega_tdt) - \cos(\theta_{t-1})) \\
		\theta_{t-1} + \omega_tdt
		\end{array}\right) = g(\mathbf{x}_{t-1}, \mathbf{u}_t)
	\]
	with Jacobians
	\begin{align*}
		G_x = &\left(\begin{array}{ccc}
		1 & 0 & \frac{V_t}{\omega_t}(\cos(\theta_{t-1} + \omega_tdt) - \cos(\theta_{t-1})) \\
		0 & 1 & \frac{V_t}{\omega_t}(\sin(\theta_{t-1} + \omega_tdt) - \sin(\theta_{t-1})) \\
		0 & 0 & 1
		\end{array}\right) \\
		G_u = \frac{1}{\omega_t}&\left(\begin{array}{cc}
		\sin(\theta_t) - \sin(\theta_{t-1}) & -\frac{V_t}{\omega_t}(\sin(\theta_t) - \sin(\theta_{t-1})) + V_t\cos(\theta_t)dt \\
		-\cos(\theta_t) + \cos(\theta_{t-1}) & \frac{V_t}{\omega_t}(\cos(\theta_t) - \cos(\theta_{t-1})) + V_t\sin(\theta_t)dt \\
		0 & \omega_tdt
		\end{array}\right).
	\end{align*}
	where $\theta_t := \theta_{t-1} + \omega_tdt$.
	\item See submitted code. Let the belief state at $t-1$ be distributed as $\mathcal{N}(\mathbf{x}_{t-1}, P_{t-1})$. We model the dynamic certainty with white noise $\nu \sim \mathcal{N}(\mathbf{0}, Q)$ applied to the control input. For a time step of $dt$, the EKF prediction step is
	\begin{align*}
		\bar{\mathbf{x}}_t &= g(\mathbf{x}_{t-1}, \mathbf{u}_t) \\
		\bar P_t &= G_xP_{t-1}G_x^T + dt \cdot G_uQG_u^T.
	\end{align*}
	\item See submitted code. Let $\mathbf{m} = (\alpha, r)$ be the polar coordinate parameters of a line in the world frame. We are given the robot's state $\mathbf{x}_{rob} = (x_{rob}, y_{rob}, \theta_{rob})$, which is the offset/yaw of the robot's base frame with respect to the world frame, and $\mathbf{x}_{cam} = (x_{cam}, y_{cam}, \theta_{cam})$, the offset/yaw of the camera's frame with respect to the robot's base frame. To transform between frames, we must first translate by the $(x_0,y_0)$ coordinates of the new origin, then rotate by the relative angle $\theta$ between the frames.
	
	Define the (counterclockwise) rotation matrix and translation function
	\[
		R(\theta) = \left(\begin{array}{cc}
			\cos(\theta) & -\sin(\theta) \\
			\sin(\theta) & \cos(\theta)
		\end{array}\right) \quad \mbox{and} \quad 
		T(\mathbf{p}; \mathbf{p}_0) = \left(\begin{array}{c}
			x + x_0 \\
			y + y_0
		\end{array}\right).
	\]
	where $\mathbf{p} = (x,y)$ and $\mathbf{p}_0 = (x_0,y_0)$. Given a point in space, the relationship between its $(x,y)$-coordinates in the old frame and $(\tilde x, \tilde y)$-coordinates in the new frame is
	\begin{align*}
		\left(\begin{array}{c}
			x \\
			y \\
		\end{array}\right) &= T(R(\theta)\mathbf{\tilde p}; \mathbf{p}_0) 
		= \left(\begin{array}{cc}
		\cos(\theta) & -\sin(\theta) \\
		\sin(\theta) & \cos(\theta)
		\end{array}\right)\left(\begin{array}{c}
			\tilde x \\
			\tilde y
		\end{array}\right) + \left(\begin{array}{c}
			x_0 \\
			y_0
		\end{array}\right) \\
		&= \left(\begin{array}{c}
			\tilde x\cos(\theta) - \tilde y\sin(\theta) \\
			\tilde x\sin(\theta) + \tilde y\cos(\theta)
		\end{array}\right) + \left(\begin{array}{c}
		x_0 \\
		y_0
		\end{array}\right) \\
		\left(\begin{array}{c}
			\tilde x \\
			\tilde y
		\end{array}\right) &= R(-\theta)T(\mathbf{p}; -\mathbf{p}_0)
		= \left(\begin{array}{cc}
		\cos(\theta) & \sin(\theta) \\
		-\sin(\theta) & \cos(\theta)
		\end{array}\right)\left(\begin{array}{c}
		x - x_0 \\
		y - y_0
		\end{array}\right) \\
		&= \left(\begin{array}{c}
			x\cos(\theta) + y\sin(\theta) \\
			-x\sin(\theta) + y\cos(\theta)
		\end{array}\right) + \left(\begin{array}{c}
			x_0\cos(\theta) + y_0\sin(\theta) \\
			-x_0\sin(\theta) + y_0\cos(\theta)
		\end{array}\right).
	\end{align*}
	If our line has polar parameters $(\alpha,r)$ in the old frame, it can be expressed as
	\begin{align*}
		x\cos(\alpha) + y\sin(\alpha) &= r \\
		(\tilde x\cos(\theta) - \tilde y\sin(\theta) + x_0)\cos(\alpha) + (\tilde x\sin(\theta) + \tilde y\cos(\theta) + y_0)\sin(\alpha) &= r \\
		\tilde x(\cos(\theta)\cos(\alpha) + \sin(\theta)\sin(\alpha)) + \tilde y(\cos(\theta)\sin(\alpha) -\sin(\theta)\cos(\alpha)) &= r - x_0\cos(\alpha) - y_0\sin(\alpha) \\
		\tilde x\cos(\alpha - \theta) + \tilde y\sin(\alpha - \theta) &= r - x_0\cos(\alpha) - y_0\sin(\alpha) \\
		\tilde x \cos(\tilde \alpha) + \tilde y\sin(\tilde \alpha) &= \tilde r
	\end{align*}
	where the parameters in the new frame are
	\begin{align*}
		\tilde \alpha = \alpha - \theta, \quad \tilde r = r - x_0\cos(\alpha) - y_0\sin(\alpha)
	\end{align*}
	Thus, to determine a map entry's coordinates in the camera frame, we first convert from the world frame to the robot's base frame,
	\begin{align*}
		\alpha_{rob} &= \alpha - \theta_{rob} \\
		\quad r_{rob} &= r - x_{rob}\cos(\alpha) - y_{rob}\sin(\alpha),
	\end{align*}
	then convert from the robot's base frame to the camera frame,
	\begin{align*}
		\alpha_{cam} &= \alpha_{rob} - \theta_{cam} = \alpha - \theta_{rob} - \theta_{cam} \\
		r_{cam} &= r_{rob} - x_{cam}\cos(\alpha_{rob}) - y_{cam}\sin(\alpha_{rob}) \\
		&= r - x_{rob}\cos(\alpha) - y_{rob}\sin(\alpha) - x_{cam}\cos(\alpha - \theta_{rob}) - y_{cam}\sin(\alpha - \theta_{rob}).
	\end{align*}
	The mean camera frame parameters are
	\[
		\mathbf{h}_t = \left(\begin{array}{c}
			\alpha - \theta - \theta_{cam} \\
			r - x\cos(\alpha) - y\sin(\alpha) - x_{cam}\cos(\alpha - \theta) - y_{cam}\sin(\alpha - \theta)
		\end{array}\right)
	\]
	and its Jacobian with respect to the belief state mean $\bar{\mathbf{x}}_t = (x,y,\theta)$ is
	\[
		H_t = \left(\begin{array}{ccc}
			0 & 0 & -1 \\
			-\cos(\alpha) & -\sin(\alpha) & -x_{cam}\sin(\alpha - \theta) + y_{cam}\cos(\alpha - \theta)
		\end{array}\right)
	\]
	\item See submitted code. Let $\mathbf{h}_t^j$ be a predicted measurement and $\mathbf{z}_t^i$ be an observation. Define the innovation to be $\mathbf{v}_t^{ij} = \mathbf{z}_t^i - \mathbf{h}_t^j$ with covariance $S_t^{ij} = H_t^j \bar P_t H_t^{j^T} + R_t^i$, where $R_t^i$ is the covariance of $\mathbf{z}_t^i$. The Mahalanobis distance is
	\[
		d_t^{ij} = \mathbf{v}_t^{ij^T} (S_t^{ij})^{-1} \mathbf{v}_t^{ij}.
	\]
	We associate each observation with the map entry with the least Mahalanobis distance. To avoid corrupted measurements that do not correspond to any map entries, we introduce a validation gate $g$ and consider only associations where $d_t^{ij} < g^2$.
	\item See submitted code. We stack the innovation vectors, covariance matrices, and Jacobians with respect to $\mathbf{x}$ into
	\[
		\mathbf{z}_t = \left(\begin{array}{c}
			\mathbf{v}_t^1 \\
			\vdots \\
			\mathbf{v}_t^K
		\end{array}\right), \quad
		R_t = \left(\begin{array}{ccc}
			R_t^1 & & 0 \\
			 & \ddots & \\
			0 & & R_t^K
		\end{array}\right), \quad
		H_t = \left(\begin{array}{c}
			H_t^1 \\
			\vdots \\
			H_t^K
		\end{array}\right).
	\]
	\item See submitted code. The standard EKF measurement correction is
	\begin{align*}
		\Sigma_t &= H_t \bar P_t H_t^T + R_t \\
		K_t &= \bar P_t H_t^T \Sigma_t^{-1} \\
		x_t &= \bar x_t + K_t \mathbf{z}_t \\
		P_t &= \bar P_t - K_t \Sigma_t K_t^T
	\end{align*}
	\item See submitted code. I inject extra uncertainty into the system by adding Gaussian noise to the scanner measurements. Specifically, the range data is perturbed by a standard normal error term subject to the minimum/maximum bounds, i.e. given a range observation $\rho_i$, the scanner returns
	\[
		\tilde \rho_i = \max(\min(\rho_i + \epsilon_i, U), L) \quad \mbox{for } \epsilon_i \sim \mathcal{N}(0,1)
	\]
	where $L =$ \verb|range_min| and $U =$ \verb|range_max| of the particular scan update. The variance of the perturbed term is approximately
	\[
		Var(\tilde \rho_i) \approx Var(\rho_i) + Var(\epsilon_i) = Var(\rho_i) + 1,
	\]
	where for mathematical simplicity, I have ignored the effect of the bounds. In this simulation, \verb|var_rho| $= 0.05 + 1 = 1.05$.
\end{enumerate}

\section{EKF Simultaneous Localization and Mapping (SLAM)}
\begin{enumerate}
	\item See submitted code. In EKF SLAM, the belief state is augmented with
	\[
		\mathbf{x}(t) = (x(t), y(t), \theta(t), \alpha^1, r^1, \ldots, \alpha^J, r^J)
	\]
	where the map features are assumed to be static in the world frame, so that $\dot \alpha^j = 0$ and $\dot r^j = 0$ for all $j = 1,\ldots,J$. We can readily extend the discrete model from Problem 1. If $\omega(t) = 0$, the state transition function is
	\[
		\mathbf{x}_t = \left(\begin{array}{c}
		x_t \\
		y_t \\
		\theta_t \\
		\alpha_t^1 \\
		r_t^1 \\
		\vdots \\
		\alpha_t^J \\
		r_t^J
		\end{array}\right) = \left(\begin{array}{c}
		x_{t-1} + V_t\cos(\theta_{t-1})dt \\
		y_{t-1} + V_t\sin(\theta_{t-1})dt \\
		\theta_{t-1} + \omega_tdt \\
		\alpha_{t-1}^1 \\
		r_{t-1}^1 \\
		\vdots \\
		\alpha_{t-1}^J \\
		r_{t-1}^J
		\end{array}\right) = g(\mathbf{x}_{t-1}, \mathbf{u}_t)
	\]
	with Jacobians $G_x = \frac{\partial g(\mathbf{x}_{t-1}, \mathbf{u}_t)}{\partial \mathbf{x}}$ and $G_u = \frac{\partial g(\mathbf{x}_{t-1}, \mathbf{u}_t)}{\partial \mathbf{u}}$ given by
	\[
		G_x = \left(\begin{array}{cccccccc}
		1 & 0 & -V_t\sin(\theta_{t-1})dt & 0 & 0 & \ldots & 0 & 0 \\
		0 & 1 & V_t\cos(\theta_{t-1})dt & 0 & 0 & \ldots & 0 & 0 \\
		0 & 0 & 1 & 0 & 0 & \ldots & 0 & 0 \\
		0 & 0 & 0 & 1 & 0 & \ldots & 0 & 0 \\
		0 & 0 & 0 & 0 & 1 & \ldots & 0 & 0 \\
		\vdots & \vdots & \vdots & \vdots & \vdots & \ddots & \vdots & \vdots \\
		0 & 0 & 0 & 0 & 0 & \ldots & 1 & 0 \\
		0 & 0 & 0 & 0 & 0 & \ldots & 0 & 1
		\end{array}\right) \quad \mbox{and} \quad 
		G_u = \left(\begin{array}{cc}
		\cos(\theta_{t-1})dt & 0 \\
		\sin(\theta_{t-1})dt & 0 \\
		0 & dt \\
		0 & 0 \\
		0 & 0 \\
		\vdots & \vdots \\
		0 & 0 \\
		0 & 0
		\end{array}\right).
	\]
	Similarly, when $\omega(t) \neq 0$, we have
	\[
		\mathbf{x}_t = \left(\begin{array}{c}
		x_t \\
		y_t \\
		\theta_t \\
		\alpha_t^1 \\
		r_t^1 \\
		\vdots \\
		\alpha_t^J \\
		r_t^J
		\end{array}\right) = \left(\begin{array}{c}
		x_{t-1} + \frac{V_t}{\omega_t}(\sin(\theta_{t-1}+ \omega_tdt) - \sin(\theta_{t-1})) \\
		y_{t-1} - \frac{V_t}{\omega_t}(\cos(\theta_{t-1} + \omega_tdt) - \cos(\theta_{t-1})) \\
		\theta_{t-1} + \omega_tdt \\
		\alpha_{t-1}^1 \\
		r_{t-1}^1 \\
		\vdots \\
		\alpha_{t-1}^J \\
		r_{t-1}^J
		\end{array}\right) = g(\mathbf{x}_{t-1}, \mathbf{u}_t)
	\]
	and Jacobians
	\begin{align*}
		G_x = &\left(\begin{array}{cccccccc}
		1 & 0 & \frac{V_t}{\omega_t}(\cos(\theta_{t-1} + \omega_tdt) - \cos(\theta_{t-1})) & 0 & 0 & \ldots & 0 & 0 \\
		0 & 1 & \frac{V_t}{\omega_t}(\sin(\theta_{t-1} + \omega_tdt) - \sin(\theta_{t-1})) & 0 & 0 & \ldots & 0 & 0 \\
		0 & 0 & 1 & 0 & 0 & \ldots & 0 & 0 \\
		0 & 0 & 0 & 1 & 0 & \ldots & 0 & 0 \\
		0 & 0 & 0 & 0 & 1 & \ldots & 0 & 0 \\
		\vdots & \vdots & \vdots & \vdots & \vdots & \ddots & \vdots & \vdots \\
		0 & 0 & 0 & 0 & 0 & \ldots & 1 & 0 \\
		0 & 0 & 0 & 0 & 0 & \ldots & 0 & 1
		\end{array}\right) \\
		G_u = \frac{1}{\omega_t}&\left(\begin{array}{cc}
		\sin(\theta_t) - \sin(\theta_{t-1}) & -\frac{V_t}{\omega_t}(\sin(\theta_t) - \sin(\theta_{t-1})) + V_t\cos(\theta_t)dt \\
		-\cos(\theta_t) + \cos(\theta_{t-1}) & \frac{V_t}{\omega_t}(\cos(\theta_t) - \cos(\theta_{t-1})) + V_t\sin(\theta_t)dt \\
		0 & \omega_tdt \\
		0 & dt \\
		0 & 0 \\
		0 & 0 \\
		\vdots & \vdots \\
		0 & 0 \\
		0 & 0
		\end{array}\right).
	\end{align*}
	where $\theta_t := \theta_{t-1} + \omega_tdt$.
	\item See submitted code. The mean camera frame parameters for $\mathbf{m}^j = (\alpha^j, r^j)$ is
	\[
		\mathbf{h}_t^j = \left(\begin{array}{c}
		\alpha^j - \theta - \theta_{cam} \\
		r^j - x\cos(\alpha^j) - y\sin(\alpha^j) - x_{cam}\cos(\alpha^j - \theta) - y_{cam}\sin(\alpha^j - \theta)
		\end{array}\right)
	\]
	and its Jacobian with respect to $\bar{\mathbf{x}}_t = (x,y,\theta,\alpha^1,r^1,\ldots, \alpha^J,r^j)$ is
	\[
		H_t^j = \left(\begin{array}{cccccc}
		A_t^j & \mathbf{0} & \ldots & B_t^j & \ldots & \mathbf{0}
		\end{array}\right) \in \mathbb{R}^{2 \times (2J + 3)}
	\]
	where the leftmost block is
	\[
		A_t^j = \left(\begin{array}{ccc}
			0 & 0 & -1 \\
			-\cos(\alpha^j) & -\sin(\alpha^j) & -x_{cam}\sin(\alpha^j - \theta) + y_{cam}\cos(\alpha^j - \theta)
		\end{array}\right) \in \mathbb{R}^{2 \times 3}
	\]
	and the center block
	\[
		B_t^j = \left(\begin{array}{cc}
			1 & 0 \\
			x\sin(\alpha^j) - y\cos(\alpha^j) + x_{cam}\sin(\alpha^j - \theta) -y_{cam}\cos(\alpha^j - \theta) & 1
		\end{array}\right) \in \mathbb{R}^{2 \times 2}
	\]
	begins at column $4 + 2(j-1) = 2(j+1)$.
	\item See submitted code.
\end{enumerate}

\section{Turtlebot Localization}
\begin{enumerate}
	\item See submitted code.
	\item See submitted code.
\end{enumerate}

\end{document}